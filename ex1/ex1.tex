\documentclass[a4paper, 11 pt, article, accentcolor=tud7b]{tudreport}

\usepackage[utf8]{inputenc}
\usepackage{amsmath}
\usepackage{longtable}

\title{CNuVS Exercise 1}
\author{Nils Rollshausen}
\subtitle{Nils Rollshausen}

\begin{document}
	\maketitle
	\section{Dijkstra's Algorithm}
	
	\begin{longtable}[c]{l|ll|ll|ll|ll|ll|ll}
		\caption{}
		\label{Dijkstra}\\
		Visited vertices    & $P_2$ & $C_2$ & $P_3$ & $C_3$ & $P_4$ & $C_4$ & $P_5$ & $C_5$ & $P_6$ & $C_6$ & $P_7$ & $C_7$ \\ \hline
		\endfirsthead
		%
		\endhead
		%
		-                   & -    & $\infty$     & -    & $\infty$     & -    & $\infty$     & -    & $\infty$     & -    & $\infty$     & -    &  $\infty$    \\
		1                   & -    & $\infty$     & -    & $\infty$     & -    & $\infty$     & 1    & 1    & -    & $\infty$     &  1    & 4    \\
		1, 5                & -    & $\infty$     & 5    & 5    & -    & $\infty$     & 1    & 1    & -    & $\infty$     & 1    & 4    \\
		1, 5, 7             & 7    & 9    & 5    & 5    & 7    & 11   & 1    & 1    & 7    & 8    & 1    & 4    \\
		1, 5, 7, 3          & 7    & 9    & 5    & 5    & 7    & 11   & 1    & 1    & 3    & 7    & 1    & 4    \\
		1, 5, 7, 3, 6       & 7    & 9    & 5    & 5    & 7    & 11   & 1    & 1    & 3    & 7    & 1    & 4    \\
		1, 5, 7, 3, 6, 2    & 7    & 9    & 5    & 5    & 2    & 10   & 1    & 1    & 3    & 7    & 1    & 4    \\
		1, 5, 7, 3, 6, 2, 4 & 7    & 9    & 5    & 5    & 2    & 10   & 1    & 1    & 3    & 7    & 1    & 4   
	\end{longtable}
	
	\section{Multiplexing}
	\subsection*{a) Internet without Multiplexing}
	Without multiplexing, every single connection to a popular site would require a dedicated wire to that destination. If, for example, Google had a total of 500 incoming wires at a datacenter, a maximum of 500 users could connect to Google at any given time.
	\subsection*{b) TDM vs FDM}
	Time-division Multiplexing (TDM) works by compressing data into a smaller timeframe, buffering those packets and interleaving packets of multiple incoming connections on a shared high-bandwidth link before reconstructing them into multiple low-bandwidth connections again. In contrast to that, Frequency-division Multiplexing (FDM) actually transmits data from both connections at the same time, though on different, non-interfering frequencies over the same medium.
	\subsection*{c) CDM and SDM}
	Code Disivion Multiplexing is roughly analogous to FDM in a digital signal. It uses non-interfering modulation codes (orthogonal vectors) for different connections to transmit multiple data streams over a single connection.
  Space Division Multiplexing refers to modifying the spatial direction of a wireless transmission (e.g. of a cellular base station) to match the physical position of the receiving device, thereby limiting the amount of noise / interference that other devices are exposed to.
\end{document}
