\documentclass[a4paper, 11 pt, article, accentcolor=tud7b]{tudreport}

\usepackage[utf8]{inputenc}
\usepackage{amsmath}

\title{CNuVS Exercise 10}
\author{Nils Rollshausen, Daniel Drodt}
\subtitle{Nils Rollshausen, Daniel Drodt}

\begin{document}
	\maketitle
	\section{M/M/1 Queues}
	$\lambda = 20, \mu = \frac{1000ms}{40ms/r} = 25$
	\subsection*{a) System time}
	$T = \frac{1}{\mu - \lambda} = \frac{1}{5} = 0.2s$
	
	\subsection*{b) Average number of requests}
	$N = \lambda \cdot T = 20 \cdot 0.2 = 4$
	
	\subsection*{c) System utilization}
	$\rho = \frac{\lambda}{\mu} = \frac{20}{25} = 0.8 = 80\%$
	
	\subsection*{d) Average idle time in 15min}
	$t_{idle} = 15 \cdot 60 \cdot (1 - \rho) = 900 \cdot 0.2 = 180s$
	
	\subsection*{e) Probability of 3 requests}
	$p_{3} = (1 - \rho) \cdot \rho^3 = 0.2 \cdot 0.8^3 = 0.1024 = 10.24\%$
	
	\section{M/M/m Queues I}
	
	$\lambda_{1} = 0.1, \lambda_{2} = 0.2, \lambda_{3} = 0.3, \mu = \frac{1}{2.5} = 0.4$
	
	\subsection*{a) Utilization}
	$$\rho_{1} = \frac{0.1}{0.4} = 0.25$$ 
	$$\rho_{2} = \frac{0.2}{0.4} = 0.5 $$
	$$\rho_{3} = \frac{0.3}{0.4} = 0.75$$
	
	\subsection*{b) Idle probability}
	$$p_{0} = (1-\rho_{1}) \cdot (1-\rho_{2}) \cdot (1-\rho_{3}) = 0.75 \cdot 0.5 \cdot 0.25 = 0.09375 = 9.375\%$$
	
	\subsection*{c) System time}
	$$T_{1} = \frac{1}{0.4 - 0.1} = \frac{10}{3} \approx 3.3333m$$
	$$T_{2} = \frac{1}{0.4 - 0.2} = 5m$$
	$$T_{3} = \frac{1}{0.4 - 0.3} = 10m$$
	
	\subsection*{d) Number of jobs}
  $$N_{1} = 0.1 \cdot T_{1} = \frac{1}{3} \approx 0.3333$$
	$$N_{2} = 0.2 \cdot T_{2} = 0.2 \cdot 5 = 1$$
	$$N_{3} = 0.3 \cdot T_{3} = 0.3 \cdot 10 = 3$$
	
	\section{M/M/m Queues II}
	
	\subsection*{a) Arrival rate and utilization}
	$$\lambda = \lambda_{1} + \lambda_{2} + \lambda_{3} = 0.1 + 0.2 + 0.3 = 0.6$$
	$$\rho = \frac{\lambda}{m \cdot \mu} = \frac{0.6}{3 \cdot 0.4} = 0.5 = 50\%$$
	
	\subsection*{b) Probability of empty system}
	\begin{align*}
	  p_{0} &= \frac{1}{(\sum_{k=0}^{m-1} \frac{(m \cdot \rho)^{k}}{k!}) + \frac{(m \cdot \rho)^{m}}{m!} \cdot \frac{1}{1 - \rho}} \\
	        &= \frac{1}{(\sum_{k=0}^{2} \frac{(3 \cdot 0.5)^{k}}{k!}) + \frac{(3 \cdot 0.5)^{3}}{6} \cdot 2} \\
	        &= \frac{1}{1 + 1.5 + \frac{9}{8} + \frac{9}{8}} \\
	        &= \frac{1}{4.75} \approx 0.2105
	\end{align*}
	
	\subsection*{c) Average system time}
	$$\delta = p_{0} \cdot \frac{(m \cdot \rho)^{m}}{m! \cdot (1-\rho)} = p_{0} \cdot \frac{1.5^3}{3} = 1.125 \cdot p_{0} \approx 0.2368 $$
	$$ T = \frac{1}{\mu} \cdot (1 + \frac{\delta}{m \cdot (1-\rho)}) = 2.5 \cdot (1 + \frac{0.2368}{1.5}) \approx 2.8947m$$
	
	\subsection*{d) Average number of jobs}
	$$N = m \cdot \rho + \frac{\rho \cdot \delta}{1 - \rho} = 1.5 + \frac{0.5 \cdot 0.2368}{0.5} = 1.5 + 0.2368 \approx 1.7368$$
	
	\section{Queueing Theory}
	
	\subsection*{a) M/M/1 Queues}
	A M/M/1 queue describes a queue with a single serving unit, where both customer interarrival times and serving time are exponentially distributed. As such, they are useful models for most simple real-world queueing scenarios.
	
	\subsection*{b) The Markov Process}
	A Markov process is a discrete stochastic process in which the probabilities for the next state depend only on the current state. Additionally, a Markov process requires that the process is memoryless (i.e. the probability of a state change does not change with the time spent in that state), homogenous, and that process times are exponentially (or geometrically) distributed.
	
	\subsection*{c) Post Office}
	A post office with four cash desks is probably best modeled by a M/M/4 queue. Even if separate queues exist for each counter, customers most likely switch queues when any desk is idle and do not queue at every counter independently. Thus, the customers act like queueing in a single queue even if there are physically separate queues for each counter.
	
	\subsection*{d) Birth-Death processes}
	Birth-Death processes are a specialization of Markov processes in which in any state, the next following state can only be an adjacent state (e.g. $k \rightarrow k+1$ or $k \rightarrow k-1$).
	\end{document}
