\documentclass[a4paper, 11 pt, article, accentcolor=tud7b]{tudreport}

\usepackage[utf8]{inputenc}
\usepackage{amsmath}
\usepackage{longtable}

\title{CNuVS Exercise 4}
\author{Nils Rollshausen, Daniel Drodt}
\subtitle{Nils Rollshausen, Daniel Drodt}

\begin{document}
	\maketitle
	\section{Dynamic Source Routing}
	\subsection*{a) Packet Types}
	Route Request packets are sent by the sender node when a new connection has to be established. They are broadcast (flooded) to the entire network, the expected result being a response packet containing a route from the sender to the destination node. \medskip \\
	Route Reply packets are the packets sent from the destination node to the sender as a response to a Route Request packet. In a non-bidirectional network, these packets may again be delivered using flooding, otherwise they can be delivered using the established route they contain. \medskip \\
	Route Error packets are sent from nodes on the route between source and destination to the source node if there is a problem with the route and the packet can not be delivered, informing the sender that a new route has to be established.
	
	\subsection*{b) Algorithm}
	
	\begin{table}[h]
	  \centering
	  \begin{tabular}{| l | l |}
	    \hline
	    Turn1  & Message \\ \hline
	    $1 \rightarrow 2$ & RREQ(1,10,\{1,2\}) \\ \hline 
	  \end{tabular}
	  \caption{Turn 1}
	\end{table}
	
	\begin{table}[h]
	  \centering
	  \begin{tabular}{| l | l |}
	    \hline
	    Turn2  & Message \\ \hline
	    $2 \rightarrow 3$ & RREQ(1,10,\{1,2,3\}) \\ \hline 
	    $2 \rightarrow 4$ & RREQ(1,10,\{1,2,4\}) \\ \hline 
	  \end{tabular}
	  \caption{Turn 2}
	\end{table}
	
	\begin{table}[h]
	  \centering
	  \begin{tabular}{| l | l |}
	    \hline
	    Turn3  & Message \\ \hline
	    $3 \rightarrow 7$ & RREQ(1,10,\{1,2,3,7\}) \\ \hline 
	    $4 \rightarrow 5$ & RREQ(1,10,\{1,2,4,5\}) \\ \hline 
	    $4 \rightarrow 6$ & RREQ(1,10,\{1,2,4,6\}) \\ \hline 
	  \end{tabular}
	  \caption{Turn 3}
	\end{table}
	
	\begin{table}[h]
	  \centering
	  \begin{tabular}{| l | l |}
	    \hline
	    Turn4  & Message \\ \hline
	    $7 \rightarrow 8$ & RREQ(1,10,\{1,2,3,7,8\}) \\ \hline 
	    $5 \rightarrow 7$ & RREQ(1,10,\{1,2,4,5,7\}) \\ \hline 
	    $6 \rightarrow 9$ & RREQ(1,10,\{1,2,4,6,9\}) \\ \hline 
	  \end{tabular}
	  \caption{Turn 4}
	\end{table}
	
	\begin{table}[h]
	  \centering
	  \begin{tabular}{| l | l |}
	    \hline
	    Turn5  & Message \\ \hline
	    $8 \rightarrow 9$ & RREQ(1,10,\{1,2,3,7,8,9\}) \\ \hline
	    $8 \rightarrow 10$ & RREQ(1,10,\{1,2,3,7,8,10\}) \\ \hline 
	    $9 \rightarrow 8$ & RREQ(1,10,\{1,2,4,6,9,8\}) \\ \hline 
	  \end{tabular}
	  \caption{Turn 5}
	\end{table}
	
	\begin{table}[h]
	  \centering
	  \begin{tabular}{| l | l |}
	    \hline
	    Node & Cache \\ \hline
	    1    & 1,10,\{1,2,3,7,8,10\}; 1,8,\{1,2,3,7,8\}; 1,7,\{1,2,3,7\}; 1,3,\{1,2,3\}\\ \hline
	    2    & 2,10,\{2,3,7,8,10\}; 2,8,\{2,3,7,8\}; 2,7,\{2,3,7\}\\ \hline
	    3    & 3,1,\{3,2,1\}; 3,10,\{3,7,8,10\}; 3,8,\{3,7,8\}\\ \hline
	    4    & 4,1,\{4,2,1\}\\ \hline
	    5    & 5,1,\{5,4,2,1\}; 5,2,\{5,4,2\}\\ \hline
	    6    & 6,1,\{6,4,2,1\}; 6,2,\{6,4,2\}\\ \hline
	    7    & 7,1,\{7,3,2,1\}; 7,2,\{7,3,2\}; 7,10,\{7,8,10\}\\ \hline
	    8    & 8,1,\{8,7,3,2,1\}; 8,2,\{8,7,3,2\}; 8,3,\{8,7,3\}\\ \hline
	    9    & 9,1,\{9,6,4,2,1\}; 9,2,\{9,6,4,2\}; 9,4,\{9,6,4\}\\ \hline
	    10   & 10,1,\{10,8,7,3,2,1\}; 10,2,\{10,8,7,3,2\}; 10,3,\{10,8,7,3\}; 10,7,\{10,8,7\}\\ \hline
	  \end{tabular}
	  \caption{Final Route Cache}
	\end{table}
	
	The response sent by node 10 is RREP(10,1,\{1,2,3,7,8,10\}).
	
	\subsection*{c) Resulting Path}
	
	Node 1 would get the path $1 \rightarrow 2 \rightarrow 3 \rightarrow 7 \rightarrow 8 \rightarrow 10$. Other possible paths such as $1 \rightarrow 2 \rightarrow 4 \rightarrow 5 \rightarrow 7 \rightarrow 8 \rightarrow 10$ or $1 \rightarrow 2 \rightarrow 4 \rightarrow 6 \rightarrow 9 \rightarrow 8 \rightarrow 10$ would be discarded at nodes 7/8 as they already received a route request for the same route earlier and thus the later request cannot possibly be the optimal path between origin and destination.
	
	\subsection*{d) Caching}
	
	Node 4 would get a response from Node 2 immediately as Node 2 has the Route $2 \rightarrow 10$ in its Cache. The resulting route would be $4 \rightarrow 2 \rightarrow 3 \rightarrow 7 \rightarrow 8 \rightarrow 10$.
	
	\subsection*{e) Considerations}
	
	The Route obtained from Node 2 is obviously not optimal ($4 \rightarrow 5 \rightarrow 7 \rightarrow 8 \rightarrow 10$ would be shorter). However the route calculation was completed much faster and with less communication overhead, which might make the less optimal routes a worthwhile tradeoff.
	
	\section{Overlay Routing}
	
	\subsection*{a) A to C}
	Path: $A \rightarrow C$ \\
	Hop Count: 3 \\
	Stretch: $3 / 3 = 1$
	
	\subsection*{b) C to E}
	Path: $C \rightarrow B \rightarrow E$ \\
	Hop Count: 9 \\
	Stretch: $9 / 3 = 3$
	
	\subsection*{c) D to E}
	Path: $D \rightarrow A \rightarrow C \rightarrow B \rightarrow E$ \\
	Hop Count: 16 \\
	Stretch: $16 / 2 = 8$
	
\end{document}
