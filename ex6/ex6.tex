\documentclass[a4paper, 11 pt, article, accentcolor=tud7b]{tudreport}

\usepackage[utf8]{inputenc}
\usepackage{amsmath}
\usepackage{placeins}

\title{CNuVS Exercise 6}
\author{Nils Rollshausen, Daniel Drodt}
\subtitle{Nils Rollshausen, Daniel Drodt}

\begin{document}
	\maketitle
	\section{Connection basics}
	\subsection*{a) Host-to-Host vs End-to-End}
	Host-to-Host connections are connections between two devices, whereas End-to-End connections refer to logical connections between a pair of processes running on the devices.
	  
	\subsection*{b) Demultiplexing}
	When multiple processes communicate with each other on two hosts, data from the processes may be multiplexed into a single logical connection between both hosts. When the receiver receives such multiplexed packets, it has to ensure that every process receives the proper data, identified by the port number. This process is referred to as demultiplexing.
	
	\subsection*{c) Connection phases}
	A connection is typically divided into three phases: Connection setup, data exchange, and connection teardown. The respective primiteves in the transport layer are T-Connect, T-Data, and T-Disconnect.
	
	\subsection*{d) Sequence numbers}
	Sequence numbers are used to detect duplicate copies of packets or packets that have been delayed by the network and arrive out-of-order. They enable the host to either discard or re-order packets as necessary. One problem is that there is only limited space available for sequence numbers, so numbers have to be re-used eventually. If the time between re-uses is too short (e.g. on very fast connections), the receiving host might mistake an old sequence number for a current one. \\
	Additionally, using sequence numbers alone, it is not possible to between a packet getting lost in transmission or simply being delivered out-or-order.
	
	\section{Ports}
	
	\subsection*{a) Purpose}
	Protocol ports allow the OS to forward received packets to the intended receiver process while also abstracting from volatile or arbitrary process IDs. The standartization of protocol ports allows remote clients to access specific services without knowledge of the concrete setup of the server host.
	
	\subsection*{b) OS and processes}
	The OS receives all packets and 'forwards' them to the intended port. A process may request allocation of a port from the OS and can then poll the OS for incoming packets on that port.
	
	\subsection*{c) Privileged port range}
	In Linux, ports between 0 and 1023 (inclusive) can only be allocated by privileged processes, while all ports above 1023 can also be used by regular processes.
	
	\subsection*{d) Disadvantages of privileged ports}
	In general, it is undesirable to run complex software (e.g. a webserver) with superuser privileges as any vulnerability in that software would grant an attacker superuser access and thus threaten the entire system. Additionally, end users might not be able to run software with superuser privileges (e.g. on a managed system where they do not have admin rights) and thus cannot use the well-known ports at all.
	
	\subsection*{e) Workarounds}
	One possible workaround to binding a privileged port to a non-root process is to start the process \textit{with} superuser privileges, bind the desired port, and then have the process revoke its own priviliges, thereby turning into a regular, unprivileged process.
	
	\section{Alternating bit protocol}
	
	\subsection*{a) Alternating bit transfer time}
	
	To send a single 32 Kib packet over the sattelite link, we need $42ms + \frac{32 Kib}{40 \cdot 2^{20} Kib/s} = 42ms + 6.104\mu s \approx 42.0061 ms$. To transfer 5.1 Mib, we need $\frac{5.1 \cdot 2^{10} Kib}{32 Kib} = 163.2 \approx 164$ packets. For each packet, we must also transfer the acknowledgement in the opposite direction. Therefore, the total transmission time is $42.0061ms \cdot 164 \cdot 2 = 13.7780s$
	
	\subsection*{b) Unacknowledged transfer time}
	If we only acknowledge the entire transmission once it is complete, the transmission time is reduced to $42ms + \frac{5.1 \cdot 2^{10} Kib}{40 \cdot 2^{20} Kib/s} + 42.0061ms \approx 84ms + 0.13061ms \approx 0.084131s$
\end{document}
